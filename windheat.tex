% If we assume that a stellar population forms
% impulsively in the distant pass with IMF $\mu(m_*)$(with minimum mass
% $m_0$ and maximum mass $m_1$), then the surviving mass fraction at any
% future time $t$ is given by
% \begin{equation}
% f(t) =\frac{ \int_{m_0}^{m_{\rm T}(t)} m_* \mu(m_*) dm_* }{ \int_{m_0}^{m_1} m_* \mu(m_*) dm_* },
% \end{equation}
% where the main sequence turnoff mass is approximately
% \begin{equation}
% m_{\rm T}(t) \approx 2.5M_\odot~ \left( \frac{t}{10^9~{\rm yr}} \right)^{-0.4}.
% \end{equation}
% For a Salpeter IMF $\mu(m_*) \propto m_*^{-2.35}$ with $m_0=0.1M_\odot$ and $m_1=100M_\odot$,
% \begin{equation}
% f_{\rm Sal}(t) = 1.098 - 0.490 \left(\frac{t}{10^{10}~{\rm yr}} \right)^{0.14}
% \end{equation}
% {\bf NCS: we should probably use a Kroupa/Chabrier IMF, but this gets the ball rolling.}
% If we approximate post-main sequence evolution as instantaneous and define $\lambda(\Mstar)$ as the fractional mass lost during all stages of stellar evolution, then the mass loss rate density
% \begin{equation}
% q(t) = \frac{\rho_*}{\bar{m}_*} \lambda(m_{\rm T}(t)) m_{\rm T}(t) \frac{df}{dt},
% \end{equation}
% where the mean stellar mass $\bar{m}_* = \int_{m_0}^{m_{\rm T}(t)} \Mstar\mu(\Mstar)d\Mstar \approx 0.3 M_\odot$.  Further approximating $\lambda(\Mstar)=0.5$, and using the Salpeter IMF once more, gives
% \begin{equation}
% q(t) = \frac{\rho_*}{10^{10}~{\rm yr}} \times 0.11 \left(\frac{t}{10^{10}~{\rm yr}} \right)^{-1.26}.
% \end{equation}
% This is a specific, time-dependent definition of $\eta(t) (=0.11(t/t_{\rm h})^{-1.26})$; if we consider different star formation scenarios (for example, continuous star formation) or different IMFs, it will change.  Once these free parameters are specified, however, we can answer an important question: do young stellar populations increase or decrease the SMBH feeding rate $\dot{M}$?  Clearly, $\eta(t)$ is larger for young stellar populations, but these stars also have high wind velocities that diminish the stagnation radius.  Crudely approximating $v_{\rm w}=75~{\rm km~s}^{-1}$ for $m_{\rm T} < 10M_{\odot}$ and $v_{\rm w}=3000~{\rm km~s}^{-1}$ for $m_{\rm T} > 10M_{\odot}$ (motivated by the transition from dust-driven wind loss on the AGB to line-driven wind loss from Wolf-Rayet stars), we can employ the relation $\dot{M} \propto \eta(t) r_{\rm s}^{2-\Gamma}\propto \eta(t) v_{\rm w}^{-4+2\Gamma}$ (where $\rho_* \propto r^{-\Gamma}$) to determine the impact of stellar ``youth'' on SMBH feeding rates.
% {\bf AG:As I previously mentioned this discussion of the eta is not
%   quite correct...}
% \begin{figure}
% \includegraphics[width=\columnwidth]{NickPlot.eps}
% \caption{\label{NickPlot} SMBH feeding rates $\dot{M}=\eta(t) \times \Mstar(r_{\rm s})$, in arbitrary units.  The green, blue, and black curves are for galaxies with $\Gamma$ values of $0.1$, $0.5$, and $0.9$, respectively.  Solid curves represent impulsive-mode star formation, while dotted curves represent continuous-mode star formation. {\bf NCS: I think these old continuous curves are wrong, need to revise}}
% \end{figure}

% In Fig. \ref{NickPlot} we plot $\dot{M}$, in arbitrary units, as a
% function of time, for three different stellar density profiles $\Gamma
% = \{1.1, 1.5, 1.9\}$ (which are normalized to have the same mass at an
% influence radius $r _{\rm soi}=10~{\rm pc}$ around a $10^7M_\odot$
% SMBH).  We parametrize the wind velocity as
% \begin{equation} \frac{v_{\rm w}}{3000 ~\rm
% km~s^{-1}}=520-495\tanh\left( \frac{t-10^{7.5}~{\rm yr}}{10^{7}~{\rm
% yr}}\right). \label{NickV1}
% \end{equation} This counts Type II SNe heating as ``winds;'' if
% instead we are in the portion of parameter space where $r_{\rm II}$ is
% very large, then we use the alternate parametrization
% \begin{equation} \frac{v_{\rm w}}{3000 ~\rm
% km~s^{-1}}=520-495\tanh\left( \frac{t-10^{7}~{\rm yr}}{10^{6.5}~{\rm
% yr}}\right), \label{NickV2}
% \end{equation} which only allows short lived Wolf-Rayet stars to
% contribute to the high-heating mode.  The ``impulsive burst'' mode of
% star formation produces large ($\sim 10$) differences between the
% three $\dot{M}$ curves at early times, when $r_{\rm s}$ is small, but
% smaller ($\sim 3$) differences at late times, when $r_{\rm s}$ is
% large.  We also plot, as dotted curves, a simple model for the
% ``continuous'' mode of star formation, where mass loss is calculated
% as $\bar{\eta} = \int\eta(t)dt/t_{\rm h} \approx 4$ and an average
% energy injection in the wind is calculated as $\bar{v_{\rm
% w}^2}=\int\eta(t)v_{\rm w}^2(t)dt/\bar{\eta} \approx (800~{\rm
% km~s}^{-1})^2$.  Interestingly, the continuous mode of star formation
% produces small differences from late-time $\dot{M}$ seen in cusp
% galaxies with impulsive star formation; however, continuous mode star
% formation decreases late-time $\dot{M}$ by an order of magnitude
% relative to impulsive star formation in core galaxies.  {\bf NCS: I
% think this old discussion of MDot in the continuous limit is wrong,
% need to revise}

The energy and mass injection from stellar winds will be the sum ofthe
contributions from main sequence and post-main sequence (PMS) stars.
For an impulsively formed stellar population of age $t$, the mass
injection rate per unit stellar mass,$\dot{\bar{m}}(t)$, and  the energy
injection rate per unit stellar mass, $\dot{\bar{e}}(t)$,  will be given by

\begin{align} 
  \dot{\bar{m}}(t) &= \frac{\Delta M(t) \mu(M_{\rm TO}(t))
    \left|\dot{M}_{\rm TO}(t)\right| + f_{\rm MS} \int_{m_0}^{m_{\rm
        T}(t)}
    \dot{m}(\Mstar, t) \mu(\Mstar) {\rm d}\Mstar }{\bar{m}_*}\\
  \dot{\bar{e}}(t) &=\dot{e}_{\rm TO}(t)+ f_{\rm MS} \int_{m_0}^{m_{\rm T}(t)}
  \frac{\vw^2(\Mstar, t) \dot{m}(\Mstar, t) \mu(\Mstar) {\rm d}\Mstar}{\bar{m}_*}.
  \label{eq:edotImp}
\end{align} 

The first terms in each expression above correspond to the
contributions from PMS stars, while the second terms correspond to the
contributions from main sequence stars. The main sequence winds are a
small fraction of the mass, and they may be not be thermalized and
mixed with the rest of the injected gas. Thus, we include a
thermalization efficiency, $ 0\le f_{\rm MS}\le 1$, in
equation~\eqref{eq:edotImp}. Throughout this paper we set $f_{\rm
  MS}=1$. 

We assume a Salpeter IMF $\mu(\Mstar)\sim M^{-2.35}$, truncated at $0.1
\Msun$ on the low mass end and at $100 \Msun$ on the high mass
end. The corresponding mean stellar mass, $\bar{m}_*$ is 0.35 $\Msun$.

For the turnoff mass, $M_{\rm TO}$, we take the following fitting
formula 

\begin{align}
\log(M_{\rm TO})=0.24 + 0.068 x^2-0.34 x+4.76 e^{-4.58 x},
\end{align}

where $x=\log(t/10^9 {\rm years})$ and $M_{\rm TO}$ is in units
$\Msun$. This fit is designed to reproduce the results in 
Table 9 of \citet{MaederMeynet:1987a} and then asymptotes to the fit
for $M_{\rm TO}$ given in equation (9) of \citet{CiottiOstriker:2007a}
for intermediate and late times ($t \gsim 10^8$ years).

For $\Delta M(t)$ we use equation (10) from \citet{CiottiOstriker:2007a}

\begin{align}
\Delta M=
\begin{cases}
0.945 M_{\rm TO}-0.503 & M_{\rm TO} < 9 \Msun\\
 M_{\rm TO}-1.4 \Msun &  M_{\rm TO} \ge 9 \Msun
\end{cases}
\end{align}

To estimate the mass loss from main sequence stars
$\dot{m}(\Mstar, t)$,
we use equation 4 \citet{SchroderCuntz:2005a}\footnote{This
  prescription is a generalization of the Reimers' mass loss law. This
expression is derived assuming the stellar wind results from the
turbulent overflow of moaterial in the chromosphere or underneath it.} {\bf AG:
  unfortunately this is not really meant for MS stars...}

\begin{align}
  \dot{m}(\Mstar)=8 \times 10^{-14} \frac{L_* r_*}{\Mstar}
  \left(\frac{T_{\rm eff}}{\rm 4000 K}\right)^{3.5}
  \left(1+\frac{g_{\odot}}{4300 g_*}\right) \Msun \pyear,\
\end{align}

where  $R_*$, $L_*$, $T_{\rm eff}$ and $g_*$ are the stellar radius,
luminosity, effective and surface gravity respectively. $g_{\odot}$ is
the stellar surface gravity. To calculate $R_*$ and $L_*$ we use the
following scaling relations (taken from Kippenhann and Weygert Figures
22.2 and 22.3).

\begin{align}
L_*=
\begin{cases}
L_{\odot} (\Mstar/\Msun)^{3.2} & \Mstar > \Msun \\
L_{\odot} (\Mstar/\Msun)^{2.5} & \Mstar \le \Msun
\end{cases}
\end{align}

\begin{align}
r_*=
\begin{cases}
R_{\odot} (\Mstar/\Msun)^{0.57} & \Mstar > \Msun \\
R_{\odot} (\Mstar/\Msun)^{0.8} & \Mstar \le \Msun
\end{cases}
\end{align}


To get a handle on $\dot{e}_{\rm TO}(t)$, we use the results from
\citet{VossDiehl+:2009a} who use a population synthesis code to
simulate the mass and energy injection into the ISM from an OB
association. For the first $\sim 10$ Myr the energy injection will
be dominated by fast Wolf-Rayet star winds. The energy injection rate
per massive star ($\Mstar>8 \Msun$), $\dot{\mathcal{E}} (t)$, from
stellar winds is plotted in the top panel of their Figure 7 and is
well fit {\bf AG: A little more detail here...} by

\begin{align}
\dot{\mathcal{E}} (t)=
\begin{cases}
  1.3 \times 10^{36} {\rm ergs/s} & t<4 \times 10^6 {\rm years}\\
  1.3  \times 10^{36} {\rm ergs/s} \left(\frac{t}{4 \times  10^6}\right)^{-3.73} & t \ge 4 \times 10^6 {\rm years}.
\end{cases}
\label{eq:voss}
\end{align}

Equation~\eqref{eq:voss} is valid only for $t \lsim 10$ Myr. For $t\gsim
10$ Myr, $\dot{e}_{\rm TO}$ will be negligible as lower mass stars
shed their gas is slow dust-driven winds during their post-main
sequence evolution {\bf AG-Nick is the preceding statement accurate},
and for our purposes the exact functional form of $\dot{e}_{\rm
  TO}(t)$ is unimportant. So we adopt Equation~\eqref{eq:voss} for all times.

 $\dot{e}_{\rm TO}(t)$ is related to $\dot{\mathcal{E}}$  via 

\begin{align}
\dot{e}_{\rm TO}(t)=f_{8} \dot{\mathcal{E}} / \bar{m}_*,
\label{eq:eto}
\end{align}

where $f_{8}$ is the fraction of the stellar population with $\Mstar>8
\Msun$. $f_8=2.6 \times 10^{-3}$ for our assumed Salpeter IMF.


We calculate the wind velocity for main sequence winds $v_w (\Mstar,
t)$ using...{\bf AG:Current just use the value for the sun~430 km/s
  replace with the real prescription we end up using.}

The effective wind velocity in the impulsive limit may then be written
as 

\begin{align}
\bar{v}_w(t)=2 \dot{\bar{e}}(t)/\dot{\bar{m}}(t)
\label{eq:vwImp}
\end{align}

We plot $\bar{v}_w$ versus time in Figure~\ref{fig:vwImp}.

\begin{figure}
\includegraphics[width=\columnwidth]{vwImp.pdf}
\caption{\label{fig:vwImp} The effective $\vw$ from stellar winds from
  a stellar population formed in a starburst $t$ years ago.}
\end{figure}

We can then use these integrated quantities to determine the effective
$V_w$ for arbitrary star formation histories. For a stellar population
with star formation rate $S(t)$ 

\begin{align} 
  \dot{M}(t) &= \int_0^t S(t_1) \dot{\bar{m}}(t-t_1){\rm
      d}t_1\\
  \dot{E}(t) &= \int_0^t S(t_1) \dot{\bar{e}}(t-t_1){\rm
      d}t_1\\
  V_w^2(t) &=2 \dot{E}(t)/\dot{M}(t)
\end{align}

\begin{figure}
\includegraphics[width=\columnwidth]{vw.pdf}
\caption{\label{NickPlot2} Effective wind velocities $V_{\rm w}$ for
different $S(t)$.  The yellow and orange curves are the Moster SF
histories, with and without Type II SNe, respectively.  The red,
purple, and blue curves are Moster SFs convolved with a $\sin^2(t)$
function normalized to $10^7$, $10^8$, and $10^9$ yr fluctuations,
respectively.  These solid curves lack SN, but the dashed purple curve
possesses it.  Effective $V_{\rm w}$ is strongly diminished when the
variability timescale is greater than $\sim$ twice the duration of
high-velocity winds.}
\end{figure}

We show $V_{\rm W}$ for different star formation histories in
Fig. \ref{NickPlot2} {\bf AG: this plot still uses the old ad hoc vw
  prescription? If so it would need to be fixed. In any case, we still
  differ in detail on the vw from stars}.  In particular, we use
Eqs. 17-20 from \citet{MosterNaab+:2013a} and the $M_{\rm BH}-M_{\rm
  halo}$ relation from \citet{BandaraCrampton+:2009a} to define $S(t)$
for particular galaxies.  It seems that ``bumpy'' SF histories
severely diminish $V_{\rm w}$ if the timescale for SF variability is a
factor of a few or more greater than the duration of high-velocity
winds (either 10 or 40 Myr).

%%% Local Variables: 
%%% mode: latex
%%% TeX-master: "ms"
%%% End: 
