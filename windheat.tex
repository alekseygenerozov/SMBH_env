
\subsection{Single Burst}

In the case of a single impulsive burst of star formation, the mass
and energy injection rate per unit stellar mass at time $t$ after the
burst are given, respectively, by
\begin{align} 
  \dot{\bar{m}}(t) &= \frac{\Delta M(t) \mu|_{M_{\rm TO}(t)}
    \left|\dot{M}_{\rm TO}(t)\right| + f_{\rm MS} \int_{m_0}^{m_{\rm
        T}(t)}
    \dot{m}(\Mstar, t) \mu|_{\Mstar} {\rm d}\Mstar }{\bar{m}_*}\\
  \dot{\bar{e}}(t) &=\dot{e}_{\rm TO}(t)+ f_{\rm MS} \int_{m_0}^{m_{\rm T}(t)}
  \frac{\vw^2(\Mstar, t) \dot{m}(\Mstar, t) \mu|_{\Mstar} {\rm d}\Mstar}{\bar{m}_*},
  \label{eq:edotImp}
\end{align} 
where the first and second terms in each expression corresponds to
contributions from main sequence (MS) stars and post-main sequence
(PMS) stars, respectively.  Here $ f_{\rm MS} < 1$ is the efficiency
with which main sequence winds thermalize their energy with the rest
of the injected gas (we hereafter assume $f_{\rm MS} = 1$) and $\mu$
is the IMF, which we take to be of the Salpeter form $\mu|_{\Mstar}
\propto M_{\star}^{-2.35}$, truncated at $m_0 = 0.1 \Msun$ on the low
mass end and at $100 \Msun$ on the high mass end, where $\bar{m}_*$ =
0.35 $\Msun$ is the corresponding mean stellar mass.  Our optimistic
assumption of 100 per cent thermalization efficiency of stellar winds
is at least partially offset by our conservative assumption to neglect
energy input from core collapse supernovae, which contribute a
comparable energy injection rate to massive stellar winds.

The quantity of mass lost in PMS winds $\Delta M(t)$ is estimated from
the expression given by \citet{CiottiOstriker:2007a} (their eq.~[10]),
\begin{align}
\Delta M=
\begin{cases}
0.945 M_{\rm TO}-0.503 & M_{\rm TO} < 9 \Msun\\
 M_{\rm TO}-1.4 \Msun &  M_{\rm TO} \ge 9 \Msun,
\end{cases}
\end{align}
where $M_{\rm TO}$ is the turn-off mass, which at time $t < t_{\rm h}$ is calculated as
\begin{align}
\log(M_{\rm TO})/M_{\odot} =0.24 + 0.068 x^2-0.34 x+4.76 e^{-4.58 x},
\end{align}
where $x=\log(t/10^9 {\rm yr})$.  This functional fit is designed
to reproduce the results of \citet{MaederMeynet:1987a} (their Table 9)
for massive stars while asymptoting to the formula provided by
\citet{CiottiOstriker:2007a} (their eq.~[9]) for intermediate and late
times ($t \gsim 10^8$ years).

The MS wind mass loss rate $\dot{m}(\Mstar, t)$ is calculated based on
the generalization of Reimer's law
\begin{align}
  \dot{m}=4 \times 10^{-13} \frac{L_* R_*}{\Mstar} \Msun \pyear,\
\end{align}
where  $R_*$, $L_*$, $T_{\rm eff}$ and $g_*$ are the stellar radius,
luminosity, effective temperature, and surface gravity, respectively, the latter normalized to its solar value $g_{\odot}$.  The stellar radius and luminosity are estimated as (\citet{Kippenhahn&Weigert90}; Figs.~22.2) 22.3)
\begin{align}
L_*=
\begin{cases}
L_{\odot} (\Mstar/\Msun)^{3.2} & \Mstar > \Msun \\
L_{\odot} (\Mstar/\Msun)^{2.5} & \Mstar \le \Msun
\end{cases}
\end{align}
\begin{align}
R_*=
\begin{cases}
R_{\odot} (\Mstar/\Msun)^{0.57} & \Mstar > \Msun \\
R_{\odot} (\Mstar/\Msun)^{0.8} & \Mstar \le \Msun
\end{cases}
\end{align}
The wind velocity of main sequence winds is assumed to equal $v_w
(\Mstar, t) =
v_{w,\odot}(M_{\star}/M_{\odot})^{1/2}(R_{\star}/R_{\odot})^{-1/2}$,
i.e.~scaling as the stellar escape velocity and normalized to the
velocity of the solar wind $v_{w,\odot} = 430$ km s$^{-1}$; this
produces an effective wind heating velocity for main sequence winds
alone of $\sim 100$ km s$^{-1}$ for $\tau_{\star} \sim t_{\rm h}$,
close to the value found by \citet{NaimanSoares-Furtado+:2013a} for
globular clusters based on a more sophisticated population synthesis
treatment.

The rate of energy input from PMS winds, $\dot{e}_{\rm TO}(t) =
f_{8}\dot{\mathcal{E}}/\bar{m}_*$, is dominated by fast Wolf-Rayet
winds and core-collapse supernovae from massive stars ($M_{\star} >
8M_{\odot}$), where $f_{8} =2.6 \times 10^{-3}$ is the fraction of the
stellar mass with $M_{\star} > 8M_{\odot}$ for our assumed Salpeter
IMF.  Here $\dot{\mathcal{E}} (t)$ is the energy injection rate per
massive star, which we estimate as
\begin{align}
\dot{\mathcal{E}} (t)=  1.3 \times 10^{36} {\rm erg\,s^{-1}}
\begin{cases}
  1 & t<4 \times 10^6 {\rm yr}\\
  \left(\frac{t}{4 \times  10^6\,{\rm yr}}\right)^{-3.73} &  4\times
  10^{6} {\rm yr}  \le t \le 10^{8} {\rm yr}
  \times 10^6 {\rm yr},\\
  0 & t > 10^8 {\rm yr}
\end{cases}
\label{eq:voss}
\end{align}
based on the results of \citet{VossDiehl+:2009a} (their Fig.~7, top
panel), who use a population synthesis code to simulate the mass and
energy injection into the ISM from an OB association.  Although
equation~\eqref{eq:voss} is valid only for $t \lsim 10$ Myr, in
practice the precise functional form of $\dot{e}_{\rm TO}(t)$ is
generally unimportant for our purposes, so we adopt
equation~\eqref{eq:voss} as being valid for all times. 


The effective wind velocity in the limit of an impulsive star
formation may then be written as
\begin{align}
\bar{v}_w(t)=2 \dot{\bar{e}}(t)/\dot{\bar{m}}(t)
\label{eq:vwImp}
\end{align}
while 
\begin{align}
\eta = \dot{\bar{m}}(t) \th
\label{eq:etaImp}
\end{align}
 Figure~\ref{fig:vwImp} shows the values of $\bar{v}_w(t)$ and $\eta(t)$ as a function of stellar age, $\tau_{\star}$.

\begin{figure}
\includegraphics[width=\columnwidth]{vwImp.pdf}
\includegraphics[width=\columnwidth]{etaImp.pdf}
\caption{\label{fig:vwImp} Effective heating rate, $v_w^{\star}$, and mass loss parameter, $\eta$ (eq.~[\ref{eq:q}]), resulting from stellar winds from a stellar population of age $\tau_{\star}$.}
\end{figure}



\subsection{Over Star Formation History}
Generalizing to an arbitrary SFH $S(t)$, the total rate of mass and energy input can be written as
\begin{align} 
  \dot{M}(t) &= \int_0^t S(t_1) \dot{\bar{m}}(t-t_1){\rm
      d}t_1 \label{eq:MDotSFH}\\
  \dot{E}(t) &= \int_0^t S(t_1) \dot{\bar{e}}(t-t_1){\rm
      d}t_1, \label{eq:EDotSFH}
\end{align}
resulting in a wind heating parameter of 
\begin{align}
  v_w^2(t) &=2 \dot{E}(t)/\dot{M}(t).
\end{align}
The mass return parameter will be
\begin{align}
\eta = \frac{\dot{M}(t)}{\int_0^t S(t_1) {\rm d}t_1} \th
\end{align}

 \begin{figure}
  \includegraphics[width=\columnwidth]{bumpy.pdf}
  \caption{\label{fig:NickPlot2} Effective wind velocities for
    nonstandard star formation histories.  The black curves shows, for
    reference, $v_{\rm w}$ calculated using the halo-averaged $S(t)$,
    and gray curves show wind heating resulting from perturbed star
    formation histories given by equation~\eqref{eq:sfrPerturbed}. In
    the top panel the star formation rate declines for a time, $\delta
    t_*=10^{7}$ years to fractions $\iota= 0.001$ (dashed), $\iota
    =0.1$ (dotted), and $\iota = 0.3$ (dot-dashed) of the halo
    averaged value. The bottom panel shows results for $\delta
    t_*=10^{8}$ years and $\iota$=0.001 (dashed) and 0.1 (dotted).}
  \end{figure}
  We estimate the stellar wind heating provided by the {\it average}
  star formation history of galaxies of a given $\Mbh$ using the
  results of \citet[eqs.~17-20]{MosterNaab+:2013a}.  Note that the
  star formation histories in \citet{MosterNaab+:2013a} are in terms
  of halo mass. For a given $\Mbh$ we assign the halo mass whose star
  formation history would produce a bulge consistent with the
  $\Mbh-M_{\rm bulge}$ relationship from \citet{McConnellMa:2013a}.  A
  slight complication occurs for the largest mass halos, where much of
  the $z=0$ stellar mass has been acquired through accretion of
  satellite halos rather than {\it in situ} star formation.  To
  accommodate this, we incorporate analytic fits for mass accretion
  histories, taken from \citet[their eqs.~21-23]{MosterNaab+:2013a},
  assuming that the age distribution of the accreted stars is equal to
  the age distribution of those formed {\it in situ}.  This assumption
  may be conservative if the primary galaxy's accretion history is
  dominated by minor mergers with younger stellar populations.  On the
  other hand, the dynamical friction inspiral time for small satellite
  galaxies is quite long, generally much greater than the $\sim
  10^7~{\rm yr}$ for which young stars can dominate the heating
  budget.  The mass of stars accreted for halo masses, $M_{\rm halo}< 3
  \times 10^{12} \Msun$, and redshifts, $z>4$, is small and is neglected.

  To find the total mass ($\Mdot'(t)$) and energy ($\dot{E}'(t)$)
  injection rates, including the contribution of accreted stars, we
  add a convolution of the specific mass and injection rates with the
  accretion history $A(t)$ to the mass and energy injection rates from
  star formed in-situ.  Thus,

\begin{align}
\Mdot '(t)&=\Mdot(t)+\int_{0}^{t} A(t_{\rm acc}) \frac{\Mdot(t_{\rm acc})}
{\int_{0}^{t_{\rm acc}} S(t_1) dt_1} {\rm dt_{ acc}}\\
\dot{E}'(t)&=\dot{E}(t)+\int_{0}^{t} A(t_{\rm acc}) \frac{\dot{E}(t_{\rm acc})}
{\int_{0}^{t_{\rm acc}} S(t') dt_1} {\rm dt_{acc}}.
\end{align}

The corresponding wind heating parameter $v_w'$ and mass return
parameter, $\eta'$, will be given by

\begin{align}
\eta'&=\frac{\Mdot'(t)}{\int_0^t S(t_1) {\rm d}t_1+\int_0^t A(t_1) {\rm
    d}t_1} \th\\
v_w'^2&=2 \frac{\dot{E}'(t)}{\Mdot'(t)}.
\end{align}

Figure \ref{fig:NickPlot2} shows how the wind heating varies as star formation histories deviate from their halo-averaged values.  In particular, we show
\begin{align} 
  \dot{\mathcal{M}}(t) &= \int_0^t \mathcal{S}(t_1) \dot{\bar{m}}(t-t_1){\rm
      d}t_1\\
  \dot{\mathcal{E}}(t) &= \int_0^t \mathcal{S}(t_1) \dot{\bar{e}}(t-t_1){\rm
      d}t_1,\\
  \mathcal{V}_w^2(t) &=2 \dot{\mathcal{E}}(t)/\dot{\mathcal{M}}(t).
\end{align}
In these equations, 
\begin{equation}
\mathcal{S}(t) = S(t) \times
\left(\frac{2}{\pi}(1-\iota)\arctan(t/\delta t_{\star}) + \iota
\right).
\label{eq:sfrPerturbed}
\end{equation}
This function convolves the recent ($z \approx 0$) halo-averaged star
formation history with local variation to give a more pessimistic
estimate for the value of $\tilde{v}_{\rm w}$.  In particular,
replacing $S(t)$ with $\mathcal{S}(t)$ reduces the recent star
formation rate to a fraction $\epsilon$ of its halo-averaged value,
and does so for a characteristic time $\delta t_{\star}$ into the
past.  As we can see in Fig. \ref{fig:NickPlot2}, this dramatically
lowers the effective wind speed when both $\delta t_{\star} \gtrsim
10^7 ~{\rm yr}$ and $\epsilon \lesssim 0.1$, but otherwise has too
modest of an effect to change the thermal stability properties of the
flow (although the location of $r_{\rm s}$ and the value of $\dot{M}$
may change significantly).


%%% Local Variables: 
%%% mode: latex
%%% TeX-master: "ms"
%%% End: 
